\section{Formulation of the Data Compression Problem}
Z-checker has plentiful algorithms and functions for assessing lossy compressors on scientific data sets.
We list the metrics that Z-checker can evaluate for assessing lossy compressors as follows.
\begin{itemize}
\item Pointwise compression error between original and reconstructed data sets, for example, absolute error and value-range-based relative error. In this report, we will present the value range of the data in the description-of-data section, and then adopt the absolute error bound (a constant such as 1E-5) to control the error in the compression.
\item Statistical compression error between original and reconstructed data sets, such as root mean squared error (RMSE), normalized RMSE (NRMSE), and peak signal-to-noise ratio (PSNR). According to the definition of NRMSE and PSNR, the smaller NRMSE, the larger the PSNR. In this report, hence, we focus on PSNR, which is compuated as follows.
\begin{equation}
PSNR = 20\cdot \log_{10}{(value\_range)} - 10\cdot \log_{10}{(MSE)}. 
\end{equation}  
where value\_range and MSE refer to data value range and the mean squared compression error respectively.
\item Distribution of compression errors refers to the probability distribution density of the compression errors.
It is an important metric for some scientific researchers require the compression errors to follow some certain distributions, such as Gaussian distribution.
\item Compression ratio (a.k.a, compression factor) is to evaluate the reduction size as a result of the compression. It is calculated by the original data size divided by the compressed data size.
\item Bit rate (bits/value) represents the amortized number of bits used to represent a data point's value after compression.
\item Rate-distortion based on statistical compression error and bit rate. It represents the distortion quality per bit of compressed storage. 
Rate refers to bit rate in bits/value. Distortion refers to the overall deviation of the data after compression and is generally assessed via PSNR.
\item Compression and decompression rate is to evaluate the processing speed.
In order to save I/O time during the execution, not only do the users hope to get a high compression factor, but the compression also has to suffer from limited compression/decompression time such that the overall execution performance can be maximized.
Compression/decompression rate refers to the amount of data to be compressed/decompressed per second, e.g., MB/s.
\item Autocorrelation of compression errors is important for assessing the degree of autocorrelation (if any) that the lossy compressors add to the original data sets.
\item Distortion of spectrum is to evaluate the distortion of the spectrum values (generated by discrete Fourier transform) between original and reconstructed data sets.
Minimizing such distortion is required by some scientist researchers.
\end{itemize}



